% @Author: Vargas Hector <vargash1>
% @Date:   Saturday, June 25th 2016, 12:52:10 am
% @Email:  vargash1@wit.edu
% @Last modified by:   vargash1
% @Last modified time: Sunday, June 26th 2016, 5:31:11 pm
\documentclass{article}
% Package that renders tables nicely
\usepackage{longtable}
% Package that has high graphics rending capability
\usepackage{pgf}
\usepackage{amsmath}
\usepackage{amssymb}
\usepackage{siunitx}
\usepackage{graphicx}
\usepackage[version=3]{mhchem}
\usepackage[margin=1.5in]{geometry}
\pagenumbering{gobble}
\graphicspath { {./images/} }
\begin{document}
    %----------------------------------------------
    % Cover Page
    % A fancy cover page with centered text
    \thispagestyle{empty}
    \begin{center}
        \includegraphics[scale=0.3]{wit_logo}\\[1cm]
        Exam Writeup using  \LaTeX\\
        Professor G.Sirokman\\
        Hector Vargas\\
        CHEM 1100-4B
    \end{center}
    \pagebreak
    %----------------------------------------------

    %%%%%%%%%%%%%%%%%%%%%%%%%%%%%%%%%%%%%%%
    %%%%    ____                   __  %%%%
    %%%%   / __/_ _____ ___ _    <  /  %%%%
    %%%%  / _/ \ \ / _ `/  ' \   / /   %%%%
    %%%% /___//_\_\\_,_/_/_/_/  /_/    %%%%
    %%%%%%%%%%%%%%%%%%%%%%%%%%%%%%%%%%%%%%%

    %----------------------------------------------
    % Question 1 Exam 1
    \begin{center}
        \textbf{Exam 1 Part 1}\\
        \textit{Atomic Structure}
    \end{center}
    \textbf{1. There are two naturally occurring types of chlorine, ${}^{35}$ Cl $(34.969 \si{\atomicmassunit})$ and ${}^{37}$ Cl $(36.966 \si{\atomicmassunit})$}

    a) Given that the atomic weight of chlorine is $63.546 \si{\atomicmassunit}$, what are the abundances of ${}^{35}$Cl and ${}^{37}$Cl?

    b) What makes ${}^{35}$Cl different from ${}^{37}$Cl?

    Both of these isotopes of chlorine differ by the number of neutrons they both have. It's important to note that the number of protons must remain the same. A different number of protons results in a different element. The difference in neutrons is what contributes to the difference in weight.

    c)Which chlorine is regular chlorine and which one is the isotope?

    The term regular would be a bit incorrect. They are both regular chlorine, what makes them 'regular' is the fact that one of their isotopes is in higher abundance.

    \pagebreak
    %----------------------------------------------

    %----------------------------------------------
    % Question 2 Exam 1
    \begin{center}
        \textbf{Exam 1 Part 1}\\
        \textit{Unit Analysis}
    \end{center}
    \textbf{2. Consider the two different ways to arrange a large sheet of atoms}

    i)How many sulfur atoms(atomic radius $.180 \si{\nano\metre}$) can you fit on a square area $2.5 \si{\micro\metre} * 2.5 \si{\micro\metre}$ in arrangement(a)?

    ii)How many sulfur atoms(atomic radius $.180 \si{\nano\metre}$) can you fit on a square area $2.5 \si{\micro\metre} * 2.5 \si{\micro\metre}$ in arrangement(a)?

    iii)Which of these two layers is denser? By what factor?

    \pagebreak
    %----------------------------------------------

    %----------------------------------------------
    % Question 3 Exam 1
    \begin{center}
        \textbf{Exam 1 Part 1}\\
        \textit{Atoms}
    \end{center}
    \textbf{3. Consider the modern model of an atom}

    a) Draw a diagram of the structure of the atom. Indicate what type of particles exist in each region of the atom.

    b)Describe the properties of each of the three particles in the atom.

    c) What is the approximate diameter of an atom?
    \pagebreak
    %----------------------------------------------

    %----------------------------------------------
    % Question 4 Exam 1
    %
    \begin{center}
        \textbf{Exam 1 Part 2}\\
        \textit{Acid/Base}
    \end{center}
    \textbf{4. Consider the following questions about acids and bases.}

    a) Note for each compound if it is\\
    \begin{enumerate}
        \item an acid or a base
        \item weak or strong
    \end{enumerate}

    \ce{NH_{3}} \qquad \ce{HCl}

    \ce{KI} \qquad \ce{Ca(OH)_{2}}

    b) Predict the product of the following reaction. Make sure the reaction is balanced.

    c) Write a net ionic equation for b).

    d) Identify the acid and the base in the following reaction(remember the Bronsted Lowry definition)
    $$\ce{CH_{3}COOH} + \ce{NH_{3}} \rightarrow \ce{CH_{3}COO^{-}} + \ce{NH_{4}^{+}}$$
    \pagebreak

    %----------------------------------------------

    %----------------------------------------------
    % Question 5 Exam 1
    %
    \begin{center}
        \textbf{Exam 1 Part 2}\\
        \textit{Reduction/Oxidation}
    \end{center}
    \textbf{5. In the following reactions}
    \begin{enumerate}
        \item Identify the oxidation state of the elements involved
        \item Indicate what element is being oxidized and what is being reduced
        \item Indicate how many electrons were transferred in the reaction
    \end{enumerate}

    a) $\ce{2 C_{2}H_{2}} + \ce{3 O_{2}} \rightarrow \ce{2CO_{2}} + \ce{2 H_{2}O}$

    b) $\ce{Cu^{2+}} + \ce{Zn} \rightarrow  \ce{Zn^{2+}} + \ce{Cu}$

    c) $\ce{P_{4}} + \ce{10 HClO} + \ce{6 H_{2}O} \rightarrow \ce{4 H_{3}PO_{4}} + \ce{10 HCl}$
    \pagebreak
    %----------------------------------------------

    %----------------------------------------------
    % Question 6 Exam 1
    %
    \begin{center}
        \textbf{Exam 1 Part 2}\\
        \textit{Galvanic Series}
    \end{center}
    \textbf{6. You are handed three metal samples. You are told that the three samples are silver, cobalt, and manganese, but that it isn't known which is which. You have available to you some zinc nitrate solution, nickel nitrate solution, and concentrated hydrochloric acid}

    Describe in detail, with specifics, how you could use the chemicals available to identify the three metal samples. Feel free to make use of the galvanic series below.

    \pagebreak
    %----------------------------------------------

    %----------------------------------------------
    % Question 7 Exam 1
    %
    \begin{center}
        \textbf{Exam 1 Part 3}\\
        \textit{Stoichiometry w/ limiting reagent}
    \end{center}
    \textbf{7. One process used in gold mines to recover gold from mined rock is as shown below}
    $$  \ce{Au} +  \ce{NaCN} +   \ce{O_{2}} +  \ce{H_{2}O} \rightarrow  \ce{NaAu(CN)_{2}} +  \ce{NaOH}$$

    a) Balance the reaction

    b) A small test batch is run in $10 \si{\milli\liter}$($10 \si{\milli\liter}$ is $10 \si{\gram}$ of water). The sample in the reaction contains $0.0178 \si{\gram}$ of gold. It is treated with $0.0115 \si{\gram}$ of $\ce{NaCN}$. $\ce{O_{2}}$ is present in excess. What is the limiting reagent in this test reaction?

    c) What mass of $\ce{NaAu(CN)_{2}}$ would be produced by the reaction described by b?

    \pagebreak
    %----------------------------------------------

    %----------------------------------------------
    % Question 8 Exam 1
    %
    \begin{center}
        \textbf{Exam 1 Part 3}\\
        \textit{Stoichiometry With Story}
    \end{center}
    \textbf{8. Back in the day, aluminum was a highly precious metal. It was first isolated in the early 1800s. It's synthesis involved the reaction of aluminum chloride with potassium}
    $$ \ce{3K} + \ce{AlCl_{3}} \rightarrow \ce{Al} + \ce{3KCl}$$

    The Baron Von Markov has decided that he wants a full aluminum place setting, and you have been pout in charge of making the required aluminum. You have to produce $520 \si{\gram}$ of aluminum using the reaction shown above, which work in $87\%$ yield. How much aluminum chloride($\ce{AlCl_{3}}$) and potassium($\ce{K}$) will you need to be able to produce the needed amount of aluminum?

    \pagebreak
    %----------------------------------------------

    %----------------------------------------------
    % Question 9 Exam 1
    %
    \begin{center}
        \textbf{Exam 1 Part 3}\\
        \textit{Stoichiometry With Solutions Chemistry}
    \end{center}
    \textbf{9. Early lighter than air balloons often used hydrogen gas for lift. This hydrogen gas was generally produced from the reaction of iron and sulfuric acid as shown here:}
    $$\ce{Fe} + \ce{H_{2}SO_{4}} \rightarrow \ce{FeSO_{4}} + \ce{H_{2}}$$

    a) Is this reaction an acid base reaction or an oxidation reduction reaction(or neither?). Justify your answer.

    b) In a test reaction, reacting $100 \si{\gram}$ of iron with excess sulfuric acid produced $1.45 \si{\mole}$ of hydrogen gas. What is the percent yield of this reaction?

    c) A balloon with the volume of $100 \si{\liter}$ needs to be filled with hydrogen. How much iron and sulfuric acid will be needed to produce the required amount of hydrogen gas to fill the $100 \si{\liter}$ balloon? To help, 1 $\si{\mole}$ of gas fills $24 \si{\liter}$ of volume. \textbf{Make sure to account for the yield calculated in b!}

    \pagebreak
    %----------------------------------------------

    %----------------------------------------------
    % Bonus Exam 1
    %
    \begin{center}
        \textbf{Exam 1 Bonus}\\
    \end{center}

    A) Describe the phenomenon due to which the region of the atmosphere we call the sky is blue at noon on a sunny day. (aka Why is the sky blue?)

    B) We have talked about two ways to categorize compounds so far, molecules, and ionic compounds. How are these two types of compounds different from each other? How does difference influence the way they behave in solution?

    \textbf{Answer}\\
    These two different types of compounds primarily differ in the method in which they bond. Ionic compounds transfer electrons, whereas molecules share electrons.

    \pagebreak
    %----------------------------------------------

    %%%%%%%%%%%%%%%%%%%%%%%%%%%%%%%%%%%%%%%
    %%%%    ____                  ____ %%%%
    %%%%   / __/_ _____ ___ _    |_  | %%%%
    %%%%  / _/ \ \ / _ `/  ' \  / __/  %%%%
    %%%% /___//_\_\\_,_/_/_/_/ /____/  %%%%
    %%%%%%%%%%%%%%%%%%%%%%%%%%%%%%%%%%%%%%%

    %----------------------------------------------
    % Question 1 Exam 2
    %
    \begin{center}
        \textbf{Exam 2 Part 1}\\
        \textit{Stoichiometry}
    \end{center}
    \textbf{1. The spores of the hive mind have just landed on a planet, and you have begun your hungry search for more raw matter to build a new hive cluster. OF special interest to you are local minerals, which are predominantly unobtanium trioxide ($\ce{UbO_{3}}$). The stars in the Koprulu sector are capable of making super heavy atoms, so unobtanium($\ce{Ub}$) has an atomic weight of $558 \dfrac{\si{\gram}}{\si{\mol}}$}

    You need to hatch some Zerglings fast to kill some of the more violent local fauna, like the human farmers over the next hill. A zergling egg requires $400 \si{\kilo\gram}$ of pure unobtanium to produce, and each egg hatches two zerglings. Carbon(readily available) and unobtanium trioxide react to form unobtanium and carbon dioxide. This reaction has a yield of $85\%$ for unobtanium.

    How many kilograms of minerals and how many kilograms of carbon do you need to make 10 zerglings?

    \pagebreak
    %----------------------------------------------

    %----------------------------------------------
    % Question 2 Exam 2
    %
    \begin{center}
        \textbf{Exam 2 Part 1}\\
        \textit{Unit Analysis}
    \end{center}
    \textbf{2. Consider the two different ways to arrange a large sheet of atoms}

    i) How many gold atoms(atomic radius $.144 \si{\nano\metre}$) can you fit on a square area $1.5 \si{\micro\metre} * 1.5 \si{\micro\metre}$ in arrangement(a)?

    ii) How many gold atoms(atomic radius $.144 \si{\nano\metre}$) can you fit on a square area $1.5 \si{\micro\metre} * 1.5 \si{\micro\metre}$ in arrangement(a)?

    iii) Calculate the density $(\dfrac{\text{mass}}{\text{volume}})$ of a single atom layer of arrangement a)

    \pagebreak
    %----------------------------------------------

    %----------------------------------------------
    % Question 3 Exam 2
    %
    \begin{center}
        \textbf{Exam 2 Part 1}\\
        \textit{Red/Ox}
    \end{center}
    \textbf{3. Jewelers have numerous tests to be able to tell whether a piece of jewelry is fake. For instance, lead is often used in making fake gold items. A jeweler is handed a gold ring, but is suspicious as to whether it is gold or not. HE uses a tool to make a tiny scratch on the surface of the ring, and drips some concentrated hydrochloric acid($\ce{H^{+}}$) on it. He then observes that bubbles are formed in the acid solution. Use the table below to explain the chemistry that was observed if the ring was made of lead.}

    Although gold is not damaged by hydrochloric acid, it can be corroded by aqua regia, a mixture of nitric acid and hydrochloric acid. Below is the reaction between gold and aqua regia. Confirm that it is a red/ox reaction. Identify what element is oxidized and what is reduced. Indicate how many electrons are transferred.
    $$\ce{Au} + \ce{HNO_{3}} + \ce{4 HCl} \rightarrow \ce{HAuCl_{4}} + \ce{2 H_{2}O} + \ce{NO}$$

    \pagebreak
    %----------------------------------------------

    %----------------------------------------------
    % Question 4 Exam 2
    %
    \begin{center}
        \textbf{Exam 2 Part 2}\\
        \textit{Photon Energies}
    \end{center}
    \textbf{4. The Imperial Star Destroyer, His Left Hand, has departed space dock and is cruising around Tatooine. It spots the Rebel Frigatte, Valorous, and opens fire. His Left Hand's turbo laser batteries fires lasers with photons of wavelength $551 \si{\nano\meter}$.}

    a) If \textit{Valorous} is hit by a $10 \si{\mega\joule}$ blast of energy from the laster fire, how many photons hit the Valorous?

    $$ h = 6.626 * 10^{-34} \si{\joule} \cdot \si{\second}$$
    $$ c = 3.00 * 10^{8} \frac{\si{\meter}}{\si{\second}} $$

    \textbf{Answer}\\
    $$ E_{\text{photon}} = \dfrac{hc}{5.51 * 10^{-7} \si{\meter}} = 3.8900196 * 10^{-19} \si{\joule} $$

    b) \textit{His Left Hand} is using a copper based laser emitter. What is happening inside the copper atoms to emit photons?

    \pagebreak
    %----------------------------------------------

    %----------------------------------------------
    % Question 5 Exam 2
    %
    \begin{center}
        \textbf{Exam 2 Part 2}\\
        \textit{Electron Configuration}
    \end{center}
    \textbf{5. Consider the following questions relating to electron configurations of atoms.}

    a) Write the electron configurations of the following elements:
    \begin{enumerate}
        \item S\\
        \textbf{Answer: }
        $[\ce{Ne}]3s^{2}3p^{4}$
        \item N\\
        \textbf{Answer: }
        $[\ce{He}]2s^{2}2p^{3}$
    \end{enumerate}

    b) Draw an energy diagram for the electron configuration of oxygen (with lines for each orbital and electrons as arrows). What ion is the most likely ion that oxygen will form? Explain based on your electron configuration.

    c) Using quantum numbers, explain why the $n = 2$ shell has an $s$ and $p$ sub shell but no $d$ or $f$ sub shells.
    \pagebreak
    %----------------------------------------------

    %----------------------------------------------
    % Question 6 Exam 2
    %
    \begin{center}
        \textbf{Exam 2 Part 2}\\
        \textit{Orbitals}
    \end{center}
    \textbf{6. Consider the following questions relating to electron configurations of atoms.}

    a)Draw a digram, showing the probability of finding the electron on the $y$ axis, and distance from the nucleus on the $x$ axis for a $2s$ orbital

    b)Draw a diagram, showing the probability of finding the electron on the $y$ axis, and distance from the nucleus on the $x$ axis for a $2p$ orbital.

    c)Draw a picture of a $2p$ orbital

    d)In your diagrams and pictures above. Indicate the locations of any nodes. How many nodes does each type of orbital have?

    e)What appears to be a common theme between $2s$ and $2p$ orbitals based on your answers in d?


    \pagebreak
    %----------------------------------------------

    %----------------------------------------------
    % Question 7 Exam 2
    %
    \begin{center}
        \textbf{Exam 2 Part 3}\\
        \textit{Atomic Size}
    \end{center}
    \textbf{7. Consider the following questions relating to atomic size and lattice energy.}

    a) Put the following sets of atoms in order of increasing size and \textbf{explain} your choice for ordering
    $$\ce{O}, \ce{N}, \ce{F}, \ce{C}$$

    b) There are two components to changes in lattice energies. The values below are an example of each. What is suggested by each trend? What changes appear to have the greater influence on lattice energy? Explain in detail.
    \begin{enumerate}
        \item $\ce{LiF}$ $1060 \dfrac{\si{\kilo\joule}}{\si{\mole}} \qquad$ $\ce{CsI}$  $600 \dfrac{\si{\kilo\joule}}{\si{\mole}}$
        \item  $\ce{LiF}$ $1060 \dfrac{\si{\kilo\joule}}{\si{\mole}} \qquad$ $\ce{MgO}$  $3900 \dfrac{\si{\kilo\joule}}{\si{\mole}}$
    \end{enumerate}
    \textbf{Answer}\\

    1. $\ce{LiF}$ has a greater lattice energy due to the size difference. The increased atomic radii of $\ce{Cs}$ and $\ce{I}$ results in less lattice energy. Lithium and Fluorine are smaller relative to Cesium and Iodine, resulting in a tighter pull from the nucleus to the electrons. Thus, $\ce{LiF}_{\text{Lattice Energy}} > \ce{CsI}_{\text{Lattice Energy}}$

    2. In contrary to the previous question. The charge of each element has more influence on the lattice energy. $\ce{Mg^{2+}}$ and $\ce{O^{2-}}$ have a greater net charge difference than $\ce{Li^{+}}$ and $\ce{F^{-}}$
    \pagebreak
    %----------------------------------------------

    %----------------------------------------------
    % Question 8 Exam 2
    %
    \begin{center}
        \textbf{Exam 2 Part 3}\\
        \textit{Bond Enthalpy}
    \end{center}
    \textbf{8. Currently the vast majority of commercially used hydrogen is generated according to the following reaction.}
    $$\ce{CH4} + \ce{H2O} \rightarrow \ce{CO2} + \ce{H2}$$
    \begin{center}
        $\ce{C-H}$ $413 \frac{\si{\kilo\joule}}{\si{\mol}} \quad$
        $\ce{C-O}$ $358 \frac{\si{\kilo\joule}}{\si{\mol}} \quad$
        $\ce{O-O}$ $146 \frac{\si{\kilo\joule}}{\si{\mol}} \quad$
        $\ce{C=O}$ $799 \frac{\si{\kilo\joule}}{\si{\mol}}$\\[.2cm]
        $\ce{O-H}$ $463 \frac{\si{\kilo\joule}}{\si{\mol}} \quad$
        $\ce{C-C}$ $348 \frac{\si{\kilo\joule}}{\si{\mol}} \quad$
        $\ce{O=O}$ $495 \frac{\si{\kilo\joule}}{\si{\mol}} \quad$
        $\ce{H-H}$ $436 \frac{\si{\kilo\joule}}{\si{\mol}}$
    \end{center}

    a) What is the enthalpy change($\delta H$) for this reaction per $\ce{CO2}$? Is energy released or absorbed?

    b) There are a variety of means of recovering energy from $\ce{H2}$, either from giving heat through combustion or electricity electrochemically from a fuel cell. What is the energy released in this reaction?
    $$\ce{H2} + \ce{O2} \rightarrow \ce{2H2O}$$

    c) Combining a) and b), do some stoichiometry to find the net energy released per $\ce{CO2}$ produced.
    \pagebreak
    %----------------------------------------------

    %----------------------------------------------
    % Question 9 Exam 2
    %
    \begin{center}
        \textbf{Exam 2 Part 3}\\
        \textit{Bond Polarity}
    \end{center}
    \textbf{8. Consider the following questions about bonding and polarity.}

    a) Order the following bonds from most to least polar. Show your calculations.
    \begin{enumerate}
        \item $\ce{F-F}$ in $\ce{F2}$
        \item $\ce{S=C}$ in $\ce{CS2}$
        \item $\ce{O-H}$ in $\ce{H2O}$
        \item $\ce{Cl-H}$ in $\ce{HCl}$
        \item $\ce{B-F}$ in $\ce{BF3}$
        \item $\ce{Na-Cl}$ in $\ce{NaCl}$
    \end{enumerate}

    b)  Categorize each of the bonds in a) as polar covalent, non polar covalent, or ionic.

    c) What does it mean for a bond to be polar? How is this different from a bond being ionic?
    \pagebreak
    %----------------------------------------------

    %----------------------------------------------
    % Bonus Exam 2
    %
    \begin{center}
        \textbf{Exam 2 Bonus}\\
    \end{center}

    A) Describe the phenomenon due to which the region of the atmosphere we call the sky is blue at noon on a sunny day. (aka Why is the sky blue?)

    B) Use Molecular Orbitals to explain whether or not $\ce{He2+}$ molecule should be able to exist.

    \textbf{Answer}\\
    $$\text{Bond Order} = \dfrac{2 - 0}{2} = 1$$
    It would be reasonable for $\ce{He2+}$ to exist. A bond order of 0 would suggest it's too unstable.

    \pagebreak
    %----------------------------------------------

    %%%%%%%%%%%%%%%%%%%%%%%%%%%%%%%%%%%%%%%
    %%%%    ____                  ____ %%%%
    %%%%   / __/_ _____ ___ _    |_  / %%%%
    %%%%  / _/ \ \ / _ `/  ' \  _/_ <  %%%%
    %%%% /___//_\_\\_,_/_/_/_/ /____/  %%%%
    %%%%%%%%%%%%%%%%%%%%%%%%%%%%%%%%%%%%%%%

    %----------------------------------------------
    % Question 1 Exam 3
    %
    $$\text{Exam 3}$$
    Question 1 Exam 3\\
    \pagebreak
    %----------------------------------------------

    %----------------------------------------------
    % Question 2 Exam 3
    %
    $$\text{Exam 3}$$
    Question 2 Exam 3
    \pagebreak
    %----------------------------------------------

    %----------------------------------------------
    % Question 3 Exam 3
    %
    $$\text{Exam 3}$$
    Question 3 Exam 3
    \pagebreak
    %----------------------------------------------

    %----------------------------------------------
    % Question 4 Exam 3
    %
    $$\text{Exam 3}$$
    Question 4 Exam 3
    \pagebreak
    %----------------------------------------------

    %----------------------------------------------
    % Question 5 Exam 3
    %
    $$\text{Exam 3}$$
    Question 5 Exam 3
    \pagebreak
    %----------------------------------------------

    %----------------------------------------------
    % Question 6 Exam 3
    %
    $$\text{Exam 3}$$
    Question 6 Exam 3
    \pagebreak
    %----------------------------------------------

    %----------------------------------------------
    % Question 7 Exam 3
    %
    $$\text{Exam 3}$$
    Question 7 Exam 3
    \pagebreak
    %----------------------------------------------

    %----------------------------------------------
    % Question 8 Exam 3
    %
    $$\text{Exam 3}$$
    Question 8 Exam 3
    \pagebreak
    %----------------------------------------------

    %----------------------------------------------
    % Question 9 Exam 3
    %
    $$\text{Exam 3}$$
    Question 9 Exam 3
    \pagebreak
    %----------------------------------------------

\end{document}
