% @Author: Hector Vargas
% @Date:   2016-05-17 17:24:46
% @Last Modified by:   Hector Vargas
% @Last Modified time: 2016-05-17 18:14:19
\documentclass{article}
\usepackage{graphicx,xlop,longtable,pgf,xparse,amsmath,amssymb, siunitx}
\graphicspath{ {./} }
\usepackage[margin=0.5in]{geometry}
\begin{document}
    a) We can use an equib. to determine what a weak acid or weak base is by making an equib. formula using the product and reactants.
    Take for example the following equation.
    $$ HF \rightleftharpoons [H^{+}] + [F^{-}]$$
    We can then form an icebox and based on the equation we get from this icebox, we can get the
    pH or pOH of the equib. Typically one could recognize an acid or base by starting with H or ending with OH
    respectively. Based on this pH or pOH reading, we can determine if the solutions is a strong acid, stong base, weak acid, or weak base.
    pH readings closer to 14 tend to be associated with strong bases. pH readings closer to 1 signify a strong acid. Since a pH is really a measure of the relative amount of free hydrogen
    and we know that a strong acid will completely dissociate in water. Then we can say that the corrleation for pH readings weak acids and weak bases is farther from
    a strong acid reading or strong base reading. Here is a diagram showcasing what the trend is like for pH readings of acids and bases.
    $$ \includegraphics[scale=0.50]{avn1a.png}$$
    b)\\
    First we need to do some calculations in order to proceed.
    $$HCl$$
    $$\dfrac{1.0008 + 35.34}{.034 \si{\gram}} = \dfrac{9.325 * 10^{-4} \si{\mol}}{380 \si{\milli\litre}}$$
    $$\downarrow$$
    $$0.00245 \si{\mol}$$
    $$HBr$$
    $$\dfrac{1.0008 + 79.40}{.091 \si{\gram}} = \dfrac{.0013\si{\mol}}{380 \si{\milli\litre}}$$
    $$\downarrow$$
    $$0.0029 \si{\mol}$$
    $$pH = -(-\log_{10} 0.00245 + 0.0029) $$
    $$\downarrow$$
    $$pH = 2.26$$
\end{document}
