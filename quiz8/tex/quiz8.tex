% @Author: Hector Vargas
% @Date:   2016-05-17 17:24:46
% @Last Modified by:   Hector Vargas
% @Last Modified time: 2016-05-17 18:14:19
\documentclass{article}
\usepackage{graphicx,xlop,longtable,pgf,xparse,amsmath,amssymb, siunitx}
\graphicspath{ {./} }
\usepackage[margin=0.5in]{geometry}
\begin{document}
    \begin{center}
        Quiz 8\\
        \textbf{Question A}\\
        \line(1,0){525}
    \end{center}$\\$
    \indent We can use an equib. to determine what a weak acid or weak base is by making an equib. formula using the product and reactants.
    Take for example the following equation.
    $$ HF \rightleftharpoons [H^{+}] + [F^{-}]$$
    \indent We can then form an icebox and based on the equation we get from this icebox, we can get the
    pH or pOH of the equib and the k value. Strong acids and bases will completely dissociate in water; on the contrary to weak acids and bases, weak acids and bases do not dissociate.
    Thus the k value we recieve from our ice box should be low. As the equib should be shifted towards the reactants.
    Typically one could recognize an acid or base by starting with H or ending with OH
    respectively. Based on this pH or pOH reading, we can determine if the solutions is a strong acid, stong base, weak acid, or weak base.
    pH readings closer to 14 tend to be associated with strong bases. pH readings closer to 1 signify a strong acid. Since a pH is really a measure of the relative amount of free hydrogen
    and we know that a strong acid will completely dissociate in water. Then we can say that the corrleation for pH readings weak acids and weak bases is farther from
    a strong acid reading or strong base reading. Here is a diagram showcasing what the trend is like for pH readings of acids and bases.
    $$ \includegraphics[scale=0.50]{avn1a.png}$$
    \begin{center}
        \textbf{Question B}\\
        \line(1,0){525}
    \end{center}
    \indent First we need to do some calculations in order to proceed.\\
    \begin{center}
        \textbf{HCl}
    \end{center}
    $$\dfrac{1.0008 + 35.34}{.034 \si{\gram}} = \dfrac{9.325 * 10^{-4} \si{\mol}}{380 \si{\milli\litre}}$$
    $$\downarrow$$
    $$0.00245 \si{\mol}$$
    \begin{center}
        \textbf{HBr}
    \end{center}
    $$\dfrac{1.0008 + 79.40}{.091 \si{\gram}} = \dfrac{.0013\si{\mol}}{380 \si{\milli\litre}}$$
    $$\downarrow$$
    $$0.0029 \si{\mol}$$
    $$pH = -\log_{10}( 0.00245 + 0.0029 )$$
    $$\downarrow$$
    $$pH = 2.26$$
    \pagebreak
    \begin{center}
        \textbf{Question C}\\
        \line(1,0){525}
    \end{center}
    $$Sr =  87.62 \si[per-mode=symbol]{\gram\per\mol} \qquad O_{2} = 32 \si[per-mode=symbol]{\gram\per\mol} \qquad H_{2} = 2\si[per-mode=symbol]{\gram\per\mol} $$
    $$\downarrow$$
    $$\dfrac{.45 \si{\gram}}{1} * \dfrac{1 \si{\mol}}{121.62 \si{\gram}} = \dfrac{.0037 \si{\mol}}{890 \si{\milli\litre}} = 0.0042 M$$
    $$\downarrow$$
    $$pH = 14 - (-\log_{10} (.0042)) = 11.62 $$
    \begin{center}
        \textbf{Question D}\\
        \line(1,0){525}\\[0.1in]
        Acid = $.003 + .002$ \qquad Base = $.0084$\\
    \end{center}
    $$ .0084 - .005 = .0034 $$
    $$ \downarrow $$
    $$pH = 14 - (-\log_{10}(.0034)) = 11.53 $$

\end{document}
